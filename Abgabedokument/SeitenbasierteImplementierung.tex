\chapter{Seitenbasierte Implementierung --- Überblick}
\section{Schnittstelle als Repository Bean}
Ablauf der Implementierung
\begin{enumerate}
  \item Erstellen der Klasse RedisConfiguration. \\
  Dort wird alles initialisiert, was für die Verbindung mit Redis notwendig ist, z. B. Host, Port, Passwort und das Redis Template.
  \item Erstellen der Klasse Persistency. \\
  Dies ist die Schnittstelle zwischen der Redis Datenbank und der Java Implementierung. \\
  Folgende Funktionen werden bereitgestellt:
  \begin{itemize}
    \item Check, ob der User bereits in der Datenbank existiert.
    \item Alle User in einer \texttt{List} anzeigen lassen.
    \item Einen User der Datenbank hinzufügen.
    \item Einen User anhand desselben Users aus der Datenbank zurückgeben.
    \item Einen User anhand seines Usernames aus der Datenbank zurückgeben.
    \item Einen Post in die Datenbank eintragen.
    \item Einen User anhand einer \texttt{postId} finden und ihn zurückgeben.
    \item Alle Posts eines Users anhand seines Usernamen zurückgeben.
    \item Alle Posts aus der Datenbank zurückgeben lassen.
    \item Einen Post anhand einer \texttt{Id} zurückgeben lassen.
  \end{itemize}
\end{enumerate}

\newpage
\section{Sessionverwaltung}
\subsection{Registrierung}
Der User kann sich registrieren, indem er einen Username und ein Passwort auswählt. Zur Sicherheit wird er aufgefordert, das Passwort zu wiederholen. \\
Falls das wiederholte Passwort nicht mit dem zuvoreingebenen übereinstimmt, wird eine Fehlermeldung für den User ausgegeben und er wird neu auf die \texttt{Registrieren} Seite navigiert.
Bei erfolgreicher Registrierung wird der User in die Datenbank eingetragen.

\subsection{Login}
Der User loggt sich mit seinem Username und Passwort ein. \\
Fehlerbehandlung:
\begin{itemize}
  \item Wenn ein nicht registrierter User versucht sich anzumelden, wird eine Meldung ausgegeben, dass dieser nicht existiert.
  \item Wenn ein registrierter User das falsche Passwort eingibt, wird ihm angezeigt, dass das Passwort falsch eingetragen wurde.
\end{itemize}
In beiden Fällen bleibt der User auf der \texttt{Login} Seite.
Wenn der User sich erfolgreich einloggen konnte, wird eine neue Session angelegt.

\subsection{Logout}
Es gibt zwei Möglichkeiten des \texttt{Logouts}:
\begin{enumerate}
  \item Der User drückt aktiv auf die Schaltfläche \texttt{Logout}.
  \item Der User wird automatisch nach 15 Minuten Inaktivität ausgeloggt.
\end{enumerate}
In beiden Fällen wird die Session geschlossen und der User wird auf die Seite \texttt{Login} navigiert und damit zum erneuten Anmelden aufgefordert.

\newpage
\section{Implementierung weiterer Features}
\subsection{Anzeige von Nutzerprofilen und Timelines}
\subsection{Post schreiben}
\subsection{Folgen und Entfolgen}
\subsection{Suche von Nutzern}
