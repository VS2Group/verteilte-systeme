\chapter{Seitenbasierte Implementierung --- Überblick}
\section{Schnittstelle als Repository Bean}
Ablauf der Implementierung
\begin{enumerate}
  \item Erstellen der Klasse RedisConfiguration. \\
  Dort wird alles initialisiert, was für die Verbindung mit Redis notwendig ist, z. B. Host, Port, Passwort und das Redis Template.
  \item Erstellen der Klasse Persistency. \\
  Dies ist die Schnittstelle zwischen der Redis Datenbank und der Java Implementierung. \\
  Folgende Funktionen werden bereitgestellt:
  \begin{itemize}
    \item Check, ob der User bereits in der Datenbank existiert.
    \item Alle User in einer \texttt{List} anzeigen lassen.
    \item Einen User der Datenbank hinzufügen.
    \item Einen User anhand desselben Users aus der Datenbank zurückgeben.
    \item Einen User anhand seines Usernames aus der Datenbank zurückgeben.
    \item Einen Post in die Datenbank eintragen.
    \item Einen User anhand einer \texttt{postId} finden und ihn zurückgeben.
    \item Alle Posts eines Users anhand seines Usernamen zurückgeben.
    \item Alle Posts aus der Datenbank zurückgeben lassen.
    \item Einen Post anhand einer \texttt{Id} zurückgeben lassen.
  \end{itemize}
\end{enumerate}

\newpage
\section{Sessionverwaltung}
\subsection{Registrierung}
Der User kann sich registrieren, indem er einen Username und ein Passwort auswählt. Zur Sicherheit wird er aufgefordert, das Passwort zu wiederholen. \\
Falls das wiederholte Passwort nicht mit dem zuvoreingebenen übereinstimmt, wird eine Fehlermeldung für den User ausgegeben und er wird neu auf die \texttt{Registrieren} Seite navigiert.
Bei erfolgreicher Registrierung wird der User in die Datenbank eingetragen.

\subsection{Login}
Der User loggt sich mit seinem Username und Passwort ein. \\
Fehlerbehandlung:
\begin{itemize}
  \item Wenn ein nicht registrierter User versucht sich anzumelden, wird eine Meldung ausgegeben, dass dieser nicht existiert.
  \item Wenn ein registrierter User das falsche Passwort eingibt, wird ihm angezeigt, dass das Passwort falsch eingetragen wurde.
\end{itemize}
In beiden Fällen bleibt der User auf der \texttt{Login} Seite.
Wenn der User sich erfolgreich einloggen konnte, wird eine neue Session angelegt.

\subsection{Logout}
Es gibt zwei Möglichkeiten des \texttt{Logouts}:
\begin{enumerate}
  \item Der User drückt aktiv auf die Schaltfläche \texttt{Logout}.
  \item Der User wird automatisch nach 15 Minuten Inaktivität ausgeloggt.
\end{enumerate}
In beiden Fällen wird die Session geschlossen und der User wird auf die Seite \texttt{Login} navigiert und damit zum erneuten Anmelden aufgefordert.

\newpage
\section{Implementierung weiterer Features}
\subsection{Anzeige von Nutzerprofilen und Timelines}
Die Anzeige ist wie in den vorherigen Beispielen implementiert. Es wird ein Profilbild für jeden Nutzer angezeigt und außerdem kann man über einen Klick auf den Namen von einem Post-Autor klicken, um diesen aufzurufen.\\
Die Posts sind so sortiert, dass die jeweils neuesten Posts oben erscheinen.

Es gibt die folgenden Post-Listen:
\begin{enumerate}
\item Die \textbf{Profil-Posts} sind alle Posts eines Nutzers.
\item \textbf{Alle Posts} ist eine Übersicht aller Posts, die auf der Plattform hinterlassen wurden. Hierüber kann man interessante Posts lesen und sich mit neuen Nutzern verbinden.
\item \textbf{Mein Stream} ist eine Auflistung der Posts von Leuten, denen man folgt.
\end{enumerate}

\subsection{Post schreiben}
Um einen Post zu verfassen, klickt man in dem oberen Menü auf \texttt{Post schreiben}. Hierüber gelangt man auf eine neue Seite, wo man einen Post verfassen kann. Es ist sehr einfach gehalten, es gibt ein Textfeld, wo man den Post-Text eingibt und einen Button, den man anwählt, wenn man den Post abschicken kann. Der Post wird dann automatisch bei allen Posts angezeigt und erscheint auf dem Profil eines Nutzers.

\subsection{Folgen und Entfolgen}
Das Folgen eines Nutzers wird über den Folgen Button auf dessen Profil realisiert. Man kann das Profil betrachten und wenn einem die dort dargestellten Posts gefallen, kann man dem User folgen. Auf dem Profil wird dann auch direkt dargestellt, wie viele Benutzer dem Profilbenutzer folgen, und wie vielen Nutzern dieser folgt. Hierüber kann man auch feststellen, ob ein Profil beliebt ist, oder nicht.

\subsection{Suche von Nutzern}
Über die Suche in der Menüleiste kann man User suchen. Wir haben uns dafür entschieden, dass automatisch eine sogenannte \texttt{Fuzzy-Suche} gestartet wird; also die ersten Buchstaben eine Benutzers ausreichen, um diesen zu finden. Die Suche erfolgt über einen \texttt{GET}-Request. Im Anschluss wird die Ergebnis-Liste ausgegeben, bei der die Profile mit den Namen und Bildern angezeigt werden. Klickt man auf den Namen, kommt man auf das Profil des entsprechenden Benutzers.
