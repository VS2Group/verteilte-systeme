\chapter{Asynchrone Erweiterungen}
\section{Überblick der Implementierung}
\subsection{Push-Funktionalität zwischen Webserver und --client}
\paragraph{Implementierung mithilfe von STOMP--over--WebSocket}
Zur Realisierung der Websocket Funktionalitäten haben wir die \texttt{Sock.js} und \texttt{Stomp.js} JavaScript Libraries in unser Projekt eingebunden. Zusätzlich mussten zusätzlich noch Dependencies in Maven anpassen, sodass die entsprechenden Klassen benutzt werden können.
\begin{verbatim}
<dependency>
    <groupId>org.springframework.boot</groupId>k
    <artifactId>spring-boot-starter-websocket</artifactId>
</dependency>
<dependency>
    <groupId>org.springframework</groupId>
    <artifactId>spring-messaging</artifactId>
</dependency>
\end{verbatim}

Auf jeder Seite, die Benachrichtigungen über neue Posts anzeigen soll, wurde dann über Javascript eine Socket Verbindung aufgebaut (mittels der \texttt{connect()} Methode aus dem Beispiel) und somit wurde dann der \texttt{Subscribe}--Mechanismus implementiert.\\
Der \texttt{Publish}--Mechanismus wurde in der HTML Seite, die zum Posten von neuen Nachrichten erstellt wurde, implementiert. Hier wird ein \texttt{JSON-Objekt} vor dem Abschicken an den Server per \texttt{AJAX} geschickt. Dieser Request wird von einem speziellen Controller verarbeitet und hierüber werden dann die abonnierenden Clients informiert. Dieser Controller wird mit den folgenden \texttt{Annotations} versehen.

\begin{verbatim}
@MessageMapping("/hello")
@SendTo("/topic/greetings")
\end{verbatim}

\paragraph{Darstellung neuer Nachrichten}
Die Darstellung erfolgt über eine \texttt{Notification}, die anzeigt, wie viele neue Posts seit dem Aktualisieren eingagengen sind. Hierbei wird bei jeder ankommenden Nachricht geguckt, wie viele Nachrichten bisher eingangen ist und die Anzahl wird erhöht.

Beim Logout wird die Websocket-Verbindung getrennt und somit können keine weiteren Nachrichten empfangen werden.

%TODO: Lena, bitte hier einen Screenshot der Notification einfügen.
\subsection{Austausch von Updatenachrichten zwischen
Serverinstanzen}
\paragraph{Austausch über Redis Publish-Subscribe-Messaging}
Der Redis Pub-Sub Mechanismus implementiert ein Messaging-System, bei dem Publisher Nachrichten
über einen so genannten Channel senden können. Ein Redis Client kann eine beliebige Anzahl solcher Channels
abonnieren, um dort veröffentlichte Messages automatisch zu empfangen. Will ein Client die Nachrichten eines Channels
nicht mehr empfangen, so kann er diesen deabonnieren (Unsubscribe).
Für die Konfiguration unserer Anwendung, um diesen Mechanismus umzusetzen, werden in Spring, zusätzlich zu den Beans RedisTemplate und Connectionfactory,
die Beans ChannelTopic, MessageListener und RedisPublisher benötigt. Über die Annotation @Schedule kann festgelegt werden, in welchem
Zeitabstand neue Messages vom Publisher veröffentlicht werden.
